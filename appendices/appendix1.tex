\chapter{Case Studies}

I have chosen two electives to fulfill my 12 credit points of elective coursework: \textit{ELEC5305: Acoustics, Speech and Signal Processing}, and \textit{ELEC5307: Advanced Signal Processing with Deep Learning}. Below I have outlined how I plan to meet the learning outcomes from each unit of study.

\section{ELEC5305: Acoustics, Speech and Signal Processing}

\begin{itemize}
    \item LO1: Mastery of analytical and mathematical skills related to acoustic signal processing
    \begin{itemize}
        \item My thesis begins with gaining an in-depth understanding of traditional acoustic signal processing techniques such as Short-Time Fourier Transform, Mel spectrum, LOFAR spectra, and Constant Q transform. This is because a majority of state-of-the-art techniques in UATR today rely on these for their feature inputs. Only once I understand the traditional methods can I aim to build on these for more accurate underwater target recognition.
    \end{itemize}
    \item LO2: Proficiency in developing signal processing software
    \begin{itemize}
        \item Leveraging tools like MATLAB's Audio Processing Toolkit and Python's librosa library, I will develop and implement algorithms to solve various signal processing challenges, such as detrending spectra and visualising transformations.
    \end{itemize}
    \item LO3: Planning, designing, and reviewing signal processing systems
    \begin{itemize}
        \item I am designing an end-to-end signal processing pipeline as part of the UATR framework. This pipeline includes selecting signal representations, performing feature extraction, and applying machine learning models to classify underwater acoustic signals effectively.
    \end{itemize}
    \item LO4: Developing innovative ideas in signal processing systems
    \begin{itemize}
        \item My approach incorporates self-supervised and unsupervised learning techniques, aiming to go beyond traditional signal processing methods to create more robust systems for classifying acoustic data.
    \end{itemize}
    \item LO5: Communicating signal processing practice effectively
    \begin{itemize}
        \item Through technical reports, visualisations, and comprehensive documentation, I will communicate the outcomes of the signal processing techniques I develop, ensuring that both academic and industry audiences can understand and apply my findings.
    \end{itemize}
    \item LO6: Contributing to team-based projects
    \begin{itemize}
        \item As part of the Under Water Systems team at Thales, I will collaborate closely with colleagues, presenting weekly updates and receiving feedback on the direction of the project.
    \end{itemize}
\end{itemize}

\section{ELEC5307: Advanced Signal Processing with Deep Learning}

\begin{itemize}
    \item LO1: Using appropriate software platforms for multi-dimensional signal processing tasks
    \begin{itemize}
        \item My project involves the use of Python libraries such as TensorFlow, Keras, and Scikit-learn for deep learning, combined with MATLAB for more traditional signal processing tasks.
    \end{itemize}
    \item LO2: Applying machine learning and deep learning methods to multi-dimensional signal processing
    \begin{itemize}
        \item Throughout the project, I will design and implement various architectures including convolutional autoencoders, CNNs, U-Nets, and hybrid models like CNN-LSTM to explore their effectiveness in underwater acoustic target recognition.
    \end{itemize}
    \item LO3: Using existing machine learning and deep learning toolboxes
    \begin{itemize}
        \item As part of my research, I will make extensive use of pre-built libraries and toolboxes in both Python and MATLAB to implement the models required for multi-dimensional signal processing.
    \end{itemize}
    \item LO4: Reporting results professionally
    \begin{itemize}
        \item My research findings will be communicated through the thesis itself, as well as presentations and professional reports. These will be structured to meet both academic and professional standards. 
    \end{itemize}
\end{itemize}