\chapter{Experiments in Preprocessing Techniques}

% Preprocessing techniques are essential for enhancing signal clarity and improving the accuracy of subsequent analyses. These techniques aim to mitigate unwanted artefacts and noise that can obscure meaningful information within the signal. Among the numerous preprocessing methods developed, here we focus on two specific techniques: detrending and denoising. 

\begin{researchquestion}
How effective are preprocessing steps such as normalisation, detrending, and denoising in improving the clarity of underwater acoustic signal recordings?
\end{researchquestion}

\section{Normalisation}

Does perf

\section{Detrending}

Detrending is used to remove trends from the data that may introduce bias.

\subsection{\texorpdfstring{$\ell_1$}{l1} detrending}

Explain everything you did with l1 detrending.

\section{Denoising}

Denoising aims to eliminate noise components that can interfere with signal interpretation.

\subsection{Using image masking techniques}

HIDE \& SEEK

\subsection{Using the Noise2Noise framework}

Running an experiment where we approximate various recordings from the same ship as being multiple recordings of the same event.
