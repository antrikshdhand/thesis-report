\usepackage[
    inner=2.5cm,
    outer=2.5cm,
    top=2.5cm,
    bottom=2.5cm
]{geometry}
\pdfpagewidth=\paperwidth 
\pdfpageheight=\paperheight
\usepackage[utf8]{inputenc}

\emergencystretch=1em

%%% Bibliography %%%
\usepackage[backend=biber,style=ieee]{biblatex}
\addbibresource{references.bib}
\AtEveryBibitem{
    %\clearfield{doi}
    \clearfield{url}
    \clearfield{urlyear}
    \clearfield{urlmonth}
    \clearfield{note}
    % \clearfield{isbn}
    \clearfield{issn}
    \clearfield{eprint}
    \clearfield{eprinttype}
    \clearfield{langid}
}

%%% Image packages %%%
\usepackage{graphicx}
\graphicspath{ {./img/} }
\usepackage{subcaption} % For subfigures

%%% TikZ %%%
\usepackage{tikz}
\usetikzlibrary{shapes, arrows.meta, positioning}

% For drawing neural networks
\usepackage{neuralnetwork}
\newcommand{\xin}[2]{$x_#2$}
\newcommand{\xout}[2]{$\hat x_#2$}

%%% Tables %%%
\usepackage{booktabs}
\usepackage{multirow}
\usepackage{array}
\usepackage{rotating}

%%% Text %%%
\usepackage{xurl} % line breaks for urls
\usepackage{hyperref} % hyperlinks
\hypersetup{
    colorlinks=true,
    citecolor=blue!80,     % Citation color
    linkcolor=black,    % Regular internal links (e.g., table of contents)
    filecolor=black,    
    urlcolor=black,
    linktoc=all,        % Colors only the content links, not glossary
    % hidelinks,          % Hides color for glossary/acronyms and TOC links
    unicode=true
}
\usepackage{setspace} % line spacing
\usepackage{lipsum} % lorem ipsum paragraphs
\usepackage[svgnames]{xcolor} % colors
\usepackage[font={small},labelfont={bf},margin=12ex]{caption} % Makes bold labelled and smaller font captions.
\renewcommand{\thefootnote}{\alph{footnote}} % Lettered footnotes 

\usepackage[acronym, nonumberlist, nopostdot]{glossaries} %%% List of Abbreviations
\makeglossaries

\usepackage{titlesec}
\titleformat*{\paragraph}{\itshape\mdseries}

%%% TColorBox %%%
\usepackage{tcolorbox}
% Define a counter for research questions
\newcounter{researchquestion}
\setcounter{researchquestion}{1}
\renewcommand{\theresearchquestion}{\arabic{researchquestion}}

% Define a tcolorbox style for the research question with an auto-incremented number
\newtcolorbox[auto counter, number within=section]{researchquestion}{%
  colback=gray!10, % Light grey background
  colframe=black!90, % Slightly darker grey for the border
  fonttitle=\bfseries,
  title=Research Question \theresearchquestion\stepcounter{researchquestion},
  boxrule=0.5mm,
  width=\textwidth
}

% Poetry
\usepackage{verse}
\newcommand{\attrib}[1]{%
\nopagebreak{\raggedleft\footnotesize #1\par}}
\renewcommand{\poemtitlefont}{\normalfont\large\itshape\centering}
\usepackage{epigraph}
\setlength\epigraphrule{0pt}

%%% Math %%%
\usepackage{amsmath}
\usepackage{bm} % Bold vectors


\newcommand{\tikzcuboid}[4]{% width, height, depth, scale
\begin{tikzpicture}[scale=#4]
\foreach \x in {0,...,#1}
{   \draw (\x ,0  ,#3 ) -- (\x ,#2 ,#3 );
    \draw (\x ,#2 ,#3 ) -- (\x ,#2 ,0  );
}
\foreach \x in {0,...,#2}
{   \draw (#1 ,\x ,#3 ) -- (#1 ,\x ,0  );
    \draw (0  ,\x ,#3 ) -- (#1 ,\x ,#3 );
}
\foreach \x in {0,...,#3}
{   \draw (#1 ,0  ,\x ) -- (#1 ,#2 ,\x );
    \draw (0  ,#2 ,\x ) -- (#1 ,#2 ,\x );
}
\end{tikzpicture}
}


% %%%%%%% CODE INPUT %%%%%%%
% \usepackage{listings}
% \definecolor{backcolour}{rgb}{0.95,0.95,0.92}
% \definecolor{codegray}{rgb}{0.5,0.5,0.5}
% \definecolor{backgray}{RGB}{235, 235, 235}

% % Code input styling
% \lstset{
%     language=Python,
%     basicstyle=\linespread{1.2}\ttfamily\small,
%     numbers=left,
%     numberstyle=\tiny\color{codegray},
%     columns=fixed,
%     breaklines=true,
%     backgroundcolor=\color{backcolour},
%     frame=single,
%     showstringspaces=false,
%     commentstyle=\color{codegray},
%     literate={~} {$\sim$}{1},
%     upquote=true
% }

% % Code output styling
% \lstdefinestyle{output}{
%     basicstyle=\linespread{1.2}\ttfamily\small,
%     backgroundcolor=,
%     numbers=none,
%     frame=single,
%     breaklines=true,
% }
